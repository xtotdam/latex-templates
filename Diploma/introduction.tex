\section*{Введение}
\label{introduction}
\addcontentsline{toc}{section}{Введение}

Английский кокер-спаниель -- порода собак, выведенная искусственным путем ещё в начале девятнадцатого века. Изначально главной задачей ученых при выведении данной породы было создание идеального охотничьего пса. В 1902 году данная порода была признана официально, кроме того, на неё были установлены жёсткие стандарты, что сильно осложняло дальнейшее выведение, но на сегодняшний день требования, предъявляемые к представителям породы английский кокер-спаниель, сильно изменились. Родиной данной породы является Англия, хотя она в кратчайшие сроки распространилась по всему миру.\\

Английский кокер-спаниель -- это настоящая охотничья собака, сегодня её смело можно назвать спортивной, так как её неуёмная энергия постоянно приводит её в движение. Все её движения предельно энергичны с заметным размахом. Несмотря на повышенную общительность и природное дружелюбие, такие собаки зачастую проявляют недоверие к посторонним людям, они чувствительны к настроению человека. Игривый нрав и природная веселость делают этих собак привлекательными для заводчиков. Особо стоит отметить прекрасные нюх и зрение, которые делают этих собак хорошими охотниками. К числу недостатков данной породы можно отнести то, что требования, предъявляемые к ней стандартом, очень жесткие. Английские кокер-спаниели часто подвержены истерии. Это является не особенностью характера, а заболеванием~\cite{wiki:aks}.