%%%%%%%%%%%%%%%%%%%%%%%%%%%%%%%%%%%%%%%%%%%%%%%%%%%%%%%%%%%%%%%%%%%%%%
% LaTeX Example: Project Report
%
% Source: http://www.howtotex.com
%
% Feel free to distribute this example, but please keep the referral
% to howtotex.com
% Date: March 2011 

%%% Preamble
\documentclass[paper=a4, fontsize=12pt]{scrartcl}
\usepackage[T2A]{fontenc}
\usepackage[utf8]{inputenc}

\usepackage[russian]{babel}                                                         % English language/hyphenation
\usepackage[protrusion=true,expansion=true]{microtype}  
\usepackage[pdftex]{graphicx}   
\usepackage{url}
\usepackage{hyperref}


%%% Custom sectioning
\usepackage{sectsty}
\allsectionsfont{\centering \normalfont\scshape}


%%% Custom headers/footers (fancyhdr package)
\usepackage{fancyhdr}
\pagestyle{fancyplain}
\fancyhead{}                                            % No page header
\fancyfoot[L]{}                                         % Empty 
\fancyfoot[C]{}                                         % Empty
\fancyfoot[R]{\thepage}                                 % Pagenumbering
\renewcommand{\headrulewidth}{0pt}          % Remove header underlines
\renewcommand{\footrulewidth}{0pt}              % Remove footer underlines
\setlength{\headheight}{13.6pt}

%%% Maketitle metadata
\newcommand{\horrule}[1]{\rule{\linewidth}{#1}}     % Horizontal rule

\title{
        %\vspace{-1in}  
        \usefont{OT1}{bch}{b}{n}
        \normalfont \normalsize \textsc{Университет Рогов и Копыт имени коровы Бурёнки} \\
        \normalfont \normalsize \textsc{факультет завитых хвостиков} \\ [20pt]
        \horrule{0.5pt} \\[0.4cm]
        \huge Расчет удельной завитости хвоста собак породы английский кокер-спаниель \\
        \horrule{2pt} \\[0.3cm]
}
\author{
        \normalfont \normalsize Пётр Пятачков \\[-3pt]      
        \normalsize \today
}
\date{}


%%% Begin document
\begin{document}
\maketitle

Английский кокер-спаниель — порода собак, выведенная искусственным путем ещё в начале девятнадцатого века. Изначально главной задачей ученых при выведении данной породы было создание идеального охотничьего пса. В 1902 году данная порода была признана официально, кроме того, на неё были установлены жёсткие стандарты, что сильно осложняло дальнейшее выведение, но на сегодняшний день требования, предъявляемые к представителям породы английский кокер-спаниель, сильно изменились. Родиной данной породы является Англия, хотя она в кратчайшие сроки распространилась по всему миру.


\vfill \small Использованы материалы из различных источников, в том числе \url{https://ru.wikipedia.org/wiki/%D0%90%D0%BD%D0%B3%D0%BB%D0%B8%D0%B9%D1%81%D0%BA%D0%B8%D0%B9_%D0%BA%D0%BE%D0%BA%D0%B5%D1%80-%D1%81%D0%BF%D0%B0%D0%BD%D0%B8%D0%B5%D0%BB%D1%8C}


%%% End document
\end{document}
