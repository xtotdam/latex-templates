% based on http://www.dfcd.net/articles/latex/latex.html

\documentclass[a4paper,12pt]{report}
\usepackage{cmap}               % seems to enable search in pdf
\usepackage[T1,T2A]{fontenc}    % for correct work with cyrillic letters
\usepackage[utf8]{inputenc}
\usepackage{amssymb,amsfonts}
\usepackage[fleqn]{amsmath}
\usepackage[russian,english]{babel}
\usepackage{anyfontsize}
\usepackage{graphicx}
\usepackage{mathrsfs}           % calligraphic letters for math environment
\usepackage[usenames,dvipsnames,svgnames,table]{xcolor}

\usepackage{geometry}   % page geometry: margins
\geometry{left=2cm}
\geometry{right=1.5cm}
\geometry{top=1cm}
\geometry{bottom=2cm}

\usepackage{fancyhdr}   % must be loaded after geometry!
\setlength{\headheight}{15.2pt}
\renewcommand{\headrulewidth}{0pt}
\pagestyle{fancy}

\DeclareMathOperator{\sinc}{sinc}
\DeclareMathOperator{\const}{const}
\DeclareMathOperator{\Tr}{Tr}   %trace
\DeclareMathOperator{\Dim}{dim} %dimensions

\let\vaccent=\v % rename builtin command \v{} to \vaccent{}
\renewcommand{\v}[1]{\ensuremath{\mathbf{#1}}} % for vectors
\newcommand{\gv}[1]{\ensuremath{\mbox{\boldmath$ #1 $}}} % for vectors of Greek letters
\newcommand{\uv}[1]{\ensuremath{\mathbf{\hat{#1}}}} % for unit vector

\newcommand{\abs}[1]{\left| #1 \right|} % for absolute value
\newcommand{\norm}[1]{\lVert #1 \rVert} % for norm ||f||
\newcommand{\qnorm}[1]{\lVert #1 \rVert ^2} % for norm quadrat ||f||^2
\newcommand{\avg}[1]{\left< #1 \right>} % for average

\newcommand{\vmult}[2]{[\vec{#1}{\times}\vec{#2}]} % vector multiplication
\newcommand{\scmult}[2]{(\vec{#1}{\cdot}\vec{#2})} % scalar multiplication

\let\underdot=\d % rename builtin command \d{} to \underdot{}
\renewcommand{\d}[2]{\frac{d #1}{d #2}} % for derivatives
\newcommand{\dd}[2]{\frac{d^2 #1}{d #2^2}} % for double derivatives
\newcommand{\pd}[2]{\frac{\partial #1}{\partial #2}} % for partial derivatives
\newcommand{\pdd}[2]{\frac{\partial^2 #1}{\partial #2^2}} % for double partial derivatives
\newcommand{\pddv}[3]{\frac{\partial^2 #1}{\partial #2 \partial #3}} % for double partial derivatives, diff. variables
\newcommand{\pdc}[3]{\left( \frac{\partial #1}{\partial #2} \right)_{#3}} % for thermodynamic partial derivatives

\newcommand{\ket}[1]{\left| #1 \right>} % for Dirac kets
\newcommand{\bra}[1]{\left< #1 \right|} % for Dirac bras
\newcommand{\braket}[2]{\left< #1 \vphantom{#2} \right|\left. \!#2 \vphantom{#1} \right>} % for Dirac brackets
\newcommand{\matrixel}[3]{\left< #1 \vphantom{#2#3} \right| #2 \left| #3 \vphantom{#1#2} \right>} % for Dirac matrix elements

\newcommand{\comml}[2]{\left[ \hat{#1}, #2 \right]} % [A,...]
\newcommand{\commr}[2]{\left[ #1, \hat{#2} \right]} % [...,B]
\newcommand{\comm}[2]{\left[ \hat{#1}, \hat{#2} \right]} % for commutator for only two operators like [A,B]
\newcommand{\commi}[4]{\left[ \hat{#1}_{#2}, \hat{#3}_{#4} \right]} % for commutator for only two operators like [A,B] with indices

\let\divsymb=\div % rename builtin command \div to \divsymb
\renewcommand{\div}[1]{\operatorname{div}\vec{#1}} % dirty hack for divergence
\newcommand{\rot}[1]{\operatorname{rot}\vec{#1}}
\newcommand{\grad}[1]{\operatorname{grad}\vec{#1}}

\newcommand{\gradn}[1]{\vec{\nabla} #1} % for gradient
\newcommand{\divn}[1]{\scmult{\nabla}{#1}} % for divergence
\newcommand{\rotn}[1]{\vmult{\nabla}{#1}} % for curl

\newcommand{\CCBYSA}[0]{\textcircled{\scalebox{.5}{CC}} \textcircled{\scalebox{.5}{BY}} \textcircled{\scalebox{.5}{SA}}}
\newcommand{\MIT}[0]{MIT License}

\usepackage{scrtime}
\newcommand{\praiseme}[1]{\scalebox{.4}{\MIT,~\copyright~#1, compiled \the\day.\the\month.\the\year\textunderscore\thistime}}

\newcommand{\tobewritten}[0]{\textcolor{red}{\textsc{To be written}}}
\newcommand{\NB}[0]{\textbf{NB!}}
\newcommand{\Def}[0]{$\mathfrak{Def.}$}
\newcommand{\Th}[0]{$\mathfrak{[Th.]}$}
\newcommand{\Wiki}[0]{$\mathfrak{(Wiki)}$}

\newcommand{\bbar}[1]{\bar{\bar{#1}}}

\usepackage[inline]{enumitem}
\makeatletter
\newcommand{\inlineitem}[1][]{
    \ifnum\enit@type=\tw@
        {\descriptionlabel{#1}}
        \hspace{\labelsep}
    \else
        \ifnum\enit@type=\z@
        \refstepcounter{\@listctr}\fi
    \quad\@itemlabel\hspace{\labelsep}%
    \fi
}
\makeatother
