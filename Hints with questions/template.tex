\RequirePackage{snapshot}
\documentclass[11pt]{report}    % may be useful to add 'twocolumn'
\usepackage{cmap}
\usepackage[T1,T2A]{fontenc}
\usepackage[utf8]{inputenc}
\usepackage{amssymb,amsfonts}
\usepackage[fleqn]{amsmath}
\usepackage{mathrsfs}
\usepackage[russian]{babel}
\usepackage[pdftex]{graphicx}
\usepackage{url}
\usepackage{hyperref}
\usepackage[usenames,dvipsnames,svgnames,table]{xcolor}

\usepackage{enumitem}
\setlist[itemize]{itemsep=-10pt,nolistsep}

\usepackage{chemfig}
\setatomsep{25pt}

\newcommand{\chck}[0]{\textcolor{red}{\textsc{\textbf{check!}}\qquad}}
\newcommand{\idk}[0]{\textcolor{red}{\textbf{??!!}}\qquad}
\newcommand{\Wiki}[0]{$\mathfrak{(Wiki)}$}

\usepackage{scrtime}
\newcommand{\praiseme}[1]{\scalebox{.4}{ \textcircled{\scalebox{.5}{CC}} \textcircled{\scalebox{.5}{BY}} \textcircled{\scalebox{.5}{SA}} #1, \the\day.\the\month.\the\year\textunderscore\thistime*}}

\usepackage{geometry}
\geometry{left=1cm}
\geometry{right=1.5cm}
\geometry{top=1cm}
\geometry{bottom=2cm}

\usepackage{fancyhdr}
\pagestyle{fancy}
\renewcommand{\headrulewidth}{0mm}
\fancyfoot[R]{\praiseme{Xtotdam}}

\begin{document}

\begin{enumerate}
\def\labelenumi{\arabic{enumi}.}

\item \textbf{pH крови.}
\begin{itemize}
    \item 7.4
\end{itemize}

\item \textbf{Амплитуда колебаний атомов в белках.}
\begin{itemize}
    \item \chck 1.5 \AA
\end{itemize}

\item \textbf{Ароматические аминокислоты.}
\begin{itemize}
    \item Фенилаланин, тирозин, триптофан
\end{itemize}

\item \textbf{В каком порядке записывается последовательность нуклеотидов.}
\begin{itemize}
    \item Запись нуклеотидной последовательности производится в направлении от 5'-конца к 3'-концу.
\end{itemize}

\item \textbf{Вероятность диссоциации через pK и pH.}
\begin{itemize}
    \item $w^{\mp}=\frac1{1+10^{\pm(pK-pH)}}$
\end{itemize}

\item \textbf{\ldots}
\begin{itemize}
    \item \ldots
\end{itemize}

\item \textbf{Энергия изменения заряда группы.}
\begin{itemize}
    \item 2 $\div$ 3 ккал/моль
\end{itemize}

\item \textbf{Энергия одиночной ковалентной связи.}
\begin{itemize}
    \item 40 $\div$ 400 кДж/моль \Wiki
\end{itemize}

\item \textbf{Энергия проникновения заряда в неполярную среду.}
\begin{itemize}
    \item 20 $\div$ 30 ккал/моль
\end{itemize}

\item \textbf{Энергия, выделяющаяся при гидролизе одной молекулы АТФ.}
\begin{itemize}
    \item 30 кДж/моль (40 $\div$ 60 кДж/моль \Wiki)
\end{itemize}


\end{enumerate}
\end{document}
