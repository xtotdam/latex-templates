\RequirePackage{snapshot}
\documentclass[11pt,twocolumn]{report}
\usepackage{cmap}
\usepackage[T1,T2A]{fontenc}
\usepackage[utf8]{inputenc}
\usepackage{amssymb,amsfonts}
\usepackage[fleqn]{amsmath}
\usepackage{mathrsfs}
\usepackage[russian]{babel}
\usepackage[pdftex]{graphicx}
\usepackage{url}
\usepackage{hyperref}
\usepackage[usenames,dvipsnames,svgnames,table]{xcolor}
\usepackage{marvosym}
\usepackage{tikz}

\usepackage{enumitem}
\setlist[itemize]{itemsep=-10pt,nolistsep}

\usepackage{chemfig}
\setatomsep{25pt}

\newcommand{\chck}[0]{\textcolor{red}{\textsc{\textbf{check!}}\qquad}}
\newcommand{\idk}[0]{\textcolor{red}{\textbf{??!!}}\qquad}
\newcommand{\Wiki}[0]{$\mathfrak{(Wiki)}$}

\newcommand{\CCBYSA}[0]{\textcircled{\scalebox{.5}{CC}} \textcircled{\scalebox{.5}{BY}} \textcircled{\scalebox{.5}{SA}}}
\newcommand{\MIT}[0]{MIT License}

\usepackage{scrtime}
\newcommand{\praiseme}[1]{\scalebox{.4}{\MIT,~\copyright~#1, compiled \the\day.\the\month.\the\year\textunderscore\thistime}}

\usepackage{geometry}
\geometry{left=1cm}
\geometry{right=1.5cm}
\geometry{top=1cm}
\geometry{bottom=2cm}

\usepackage{fancyhdr}
\pagestyle{fancy}
\renewcommand{\headrulewidth}{0mm}
\fancyfoot[R]{\praiseme{Xtotdam}}

\begin{document}

\begin{enumerate}
\def\labelenumi{\arabic{enumi}.}

\item \textbf{pH крови.}
\begin{itemize}
    \item 7.4
\end{itemize}
%
% From my own LaTeX experience
% \pagebreak[0]     % please break the pa... nevermind
% \pagebreak[1]     % it would be nice if break the page here kind sir?
\pagebreak[2]       % why don't you break this page here? Thank you
% \pagebreak[3]     % break the page
% \pagebreak[4]     % Break! This! Fucking! Page! Right! There!

\item \textbf{\ldots}
\begin{itemize}
    \item \ldots
\end{itemize}
\pagebreak[2]

\item \textbf{Закон Ламберта-Бугера-Бэра.}
\begin{itemize}
    \item $I(x) = I_0 e^{-\varepsilon Cx} = I_0 e^{-D}$
    \item $dI = -\varepsilon C dx$

    \begin{itemize}
        \item $\varepsilon$ -- коэффициент экстинкции
        \item $C$ -- концентрация
        \item $l$ -- толщина слоя
        \item $D = \varepsilon Cx$ -- оптическая плотность
    \end{itemize}
\end{itemize}
\pagebreak[2]

\item \textbf{Изобразите пептидную связь.}
\begin{itemize}
    % \item \includegraphics[width=0.7\textwidth]{pept.png}
    \item \chemfig{C(-[5])(=[3]O)-[0]\lewis{2:,N}(-[1])(-[7]H)}  % easter egg =)
\end{itemize}
\pagebreak[2]

\item \textbf{Как называется реакция образования дисахарида.}
\begin{itemize}
    \item Конденсация
\end{itemize}
\pagebreak[2]

\item \textbf{Как называется реакция расщепления дисахарида на  моносахариды.}

\begin{itemize}
    \item Гидролиз
\end{itemize}
\pagebreak[2]

\item \textbf{Как называется связь, которую кофермент А образует с  переносимыми группами.}

\begin{itemize}
    \item Тиольная связь
\end{itemize}
\pagebreak[2]

\item \textbf{Как называются нерастворимые молекулы крахмала.}
\begin{itemize}
    \item Амилопектин
\end{itemize}
\pagebreak[2]

\item \textbf{Как называются растворимые молекулы крахмала.}
\begin{itemize}
    \item Амилоза
\end{itemize}
\pagebreak[2]

\item \textbf{Как обозначается кофермент А.}
\begin{itemize}
    % \item \includegraphics[width=0.1\textwidth]{kA.png}
    \item
    \begin{tikzpicture}[scale=.3]
        \path[draw, line width=.8] (0,-0.3)--(0,4)--(6.5,4)--(6.5,2)--(2,2)--(2,-0.3)--cycle;
        \path[draw, line width=.8] (1,1.5) node[below]{A}--(1,3)--(5,3) node[right]{S};
    \end{tikzpicture}
\end{itemize}
\pagebreak[2]

\item \textbf{Как определяется эффективный заряд молекулярной группы.}
\begin{itemize}
    \item $q^{\mp *} = \mp ew^{\mp}$
\end{itemize}
\pagebreak[2]

\item \textbf{Какая связь образуется между мономерами олигосахаридов.}
\begin{itemize}
    \item Гликозидная связь
\end{itemize}
\pagebreak[2]

\item \textbf{Какая часть общей длины ДНК отведена геному.}
\begin{itemize}
    % \item По последним данным в геноме человека содержится $30 \div 35$ тысяч генов. Интересно, что на их долю приходится только \textbf{$\sim$3\% общей длины} молекулы ДНК. Функциональная роль остальных 97\% протяженности ДНК остается пока, в значительной степени, загадкой.
    \item $\sim$3\% общей длины
\end{itemize}
\pagebreak[2]

\item \textbf{\ldots}
\begin{itemize}
    \item \ldots
\end{itemize}
\pagebreak[2]

\item \textbf{Энергия C-C связи.}
\begin{itemize}
    \item $\sim$ 350 кДж/моль
\end{itemize}
\pagebreak[2]

\item \textbf{Энергия Ван-Дер-Ваальсовой связи.}
\begin{itemize}
    \item $\sim$ 0.5 кДж/моль
\end{itemize}
\pagebreak[2]

\item \textbf{Энергия водородной связи.}
\begin{itemize}
    \item $\sim$ 3 кДж/моль
\end{itemize}
\pagebreak[2]

\item \textbf{Энергия денатурации белка.}
\begin{itemize}
    \item $\sim$ 40 кДж/моль
\end{itemize}
\pagebreak[2]

\item \textbf{Энергия изменения заряда группы.}
\begin{itemize}
    \item 2 $\div$ 3 ккал/моль
\end{itemize}
\pagebreak[2]

\item \textbf{Энергия одиночной ковалентной связи.}
\begin{itemize}
    \item 40 $\div$ 400 кДж/моль \Wiki
\end{itemize}
\pagebreak[2]

\item \textbf{Энергия проникновения заряда в неполярную среду.}
\begin{itemize}
    \item 20 $\div$ 30 ккал/моль
\end{itemize}
\pagebreak[2]

\item \textbf{Энергия, выделяющаяся при гидролизе одной молекулы АТФ.}
\begin{itemize}
    \item 30 кДж/моль (40 $\div$ 60 кДж/моль \Wiki)
\end{itemize}
\pagebreak[2]


\end{enumerate}
\end{document}
