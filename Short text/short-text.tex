\documentclass[12pt]{article}
\usepackage{amsmath}
\usepackage{fancyhdr}
\usepackage[T1,T2A]{fontenc}
\usepackage[utf8x]{inputenc}
\usepackage[english,russian]{babel}

\usepackage{geometry}
\geometry{left=2cm}
\geometry{right=1.5cm}
\geometry{top=2.8cm}
\geometry{bottom=1cm}

\usepackage{datetime}
\newdate{date}{14}{10}{2015}
\linespread{1.0}

\begin{document}

\fancyhead[L]{\displaydate{date}}
\fancyhead[R]{Author name}
\pagestyle{fancy}
\pagenumbering{gobble}

\begin{raggedright}
    \Large{\textbf{Источник образования компактных треугольных островов}}\\
    \Large\underline{\textbf{при росте металлических структур на металле}}\\
    \scriptsize{Origin of Compact Triangular Islands in Metal-on-Metal Growth}
\end{raggedright}
\\

Эпитаксиальный рост наноструктур металлов на металлических поверхностях --- сложный процесс, контролируемый огромным количеством всевозможных факторов, зависящих от разнообразных физических процессов. Он представим в виде суперпозиции элементарных актов диффузии между уже образованными на поверхности кластерами и адатомами. В частности, большое влияние на процесс оказывают межатомные взаимодействия на изгибах и углах кластеров, определяющие ход дальнейшего роста образования. \\

В проведенных исследованиях гомоэпитаксиального роста на поверхности Pt(111) с помощью СТМ было показано, что адатомы собираются в компактные треугольные острова, окруженные ступенями А-типа при температуре 400~К, обращающиеся в окруженные ступенью В-типа, и, соответственно, повернутые на 60 градусов, при нагреве до 640~К. Объяснением этому явлению может быть анизотропия межатомных взаимодействий, что подтверждается первопринципными расчетами с помощью теории функционала электронной плотности. Однако, в силу известных ограничений ТФЭП, провести расчеты на большой площади поверхности невозможно, поэтому для получения хотя бы качественных результатов используется кинетический метод Монте-Карло. \\

В данном письме демонстрируются следующие интересные эффекты, теоретически обнаруженные методом kMC. Моделирование, проведенное на поверхности $800\times 800$ атомов, показывает, что острова адатомов имеют ярко выраженную треугольную форму, стороны которых ограничены ступенями А-типа. Подобное поведение показывает общую энергетическую выгодность подобных формирований перед ступенями В-типа. При уравнивании энергий взаимодействия со ступенями обоих типов ожидаемо идет формирование шестиугольных островов, что подтверждает наблюдения. \\

При понижении температуры до 80~К вместо треугольных островов образуются кластеры дендритной формы, причем толщина "веток" определяется температурой. При данных условиях тепловые движения атомов теряют приоритет перед межатомными и атом, грубо говоря, "остается там, где сразу сел". Подобный эффект наблюдается у многих биметаллических систем. \\

В свою очередь, при обращении направления анизотропии при данной температуре при сохранении прочих параметров, моделирование демонстрирует рост "толстых" дендритов, похожих на треугольные звезды. Это возможно объяснить, рассчитав разницу барьеров анизотропии на углах и краях кластера: при малой их разнице, повышая температуру, кластер приобретет треугольную форму, иначе в результате сложных процессов перестройки форма будет более компактной --- шестиугольной. \\

Таким образом, кинетический метод Монте-Карло --- возможная альтернатива методам \nobreak{ТФЭП} на больших расчетных ячейках, позволяющий получить физичные качественные результаты, подтверждаемые в дальнейшем экспериментами.

\end{document}
